\documentclass{beamer}
\usetheme{Dresden}
\usecolortheme{beaver}
\usepackage{animate}
\usepackage[T1]{fontenc}
\usepackage[utf8]{inputenc}
\usepackage[spanish]{babel}
\usepackage{mathrsfs}
\usepackage{amsmath}
\usepackage{amssymb}
\usepackage[utf8]{inputenc}
\usepackage[T1]{fontenc}
\usepackage{multirow}
\usepackage{multicol}
\usepackage{ragged2e}
\setbeamertemplate{itemize item}{\color{red}$\blacksquare$}
\title{Lógica para ciencias de la computación}
\subtitle{Proyecto Final}
\author{Daniel Navarrete y Santiago Lopez}
\begin{document}
\begin{frame}[plain,noframenumbering]
  \titlepage 
\end{frame}


%% Diapositiva 0
\begin{frame}{Problema}
\begin{multicols}{2}
\justify Considere un numero de puntos arbitrarios. El problema consiste en encender todos los puntos sin pasar dos veces por el mismo. Un punto se enciende solo si hay una flecha que sale desde ese punto hacia otro.
\center \includegraphics[scale=0.30]{hola00.png}
\end{multicols}
\end{frame}


%% Diapositiva 0.1
\begin{frame}{Ejemplo}
\begin{multicols}{2}
\justify Por ejemplo, si unimos tres puntos de la siguiente manera, entonces todos estan  encendidos.
\center \includegraphics[scale=0.30]{hola000.png}
\end{multicols}
\end{frame}

%% Diapositiva 1
\begin{frame}
\justify Considere las letras proposicionales a,b,c,... Cada una va a representar una flecha que une unicamente dos puntos.
El siguiente diagrama ilustra la asignación para 3 puntos A,B,C.
\begin{multicols}{2}
\center \includegraphics[scale=0.30]{Tabla.png}
\\[30 pt] \justify \textbf{Filas:} Representan el punto de salida de la flecha. 
\\[20 pt] \justify \textbf{Columnas:} Representan el punto de llegada de la flecha.
\end{multicols}
\end{frame}

%% Diapositiva 2
\begin{frame}{Regla 1}
Dos flechas no pueden tener le mismo punto de partida. De acuerdo a la tabla anterior seria: \\[10 pt]
Por ejemplo: 
\begin{multicols}{2}
\center \includegraphics[scale=0.300]{Hola4.png}
\\[30 pt] \justify $$((b \wedge -c) \lor (c \wedge -b))\wedge(((d \wedge -f) $$ 
$$ \lor (f \wedge -d)) \wedge ((g \wedge -h) \lor (h \wedge -g))) $$
\end{multicols}
\end{frame}

%% Diapositiva 3
\begin{frame}{Regla 2}
A cada punto le debe llegar solo una flecha. \\[10 pt] 
Por ejemplo:
\begin{multicols}{2}
\center \includegraphics[scale=0.30]{Hola1.png}
\\[30 pt] \justify $$((d \wedge -g) \lor (g \wedge -d))\wedge(((b \wedge -h) $$ 
$$ \lor (h \wedge -b)) \wedge ((c \wedge -f) \lor (f \wedge -c))) $$
\end{multicols}
\end{frame}

%% Diapositiva 4
\begin{frame}{Regla 3}
Una vez una flecha salga de un punto B a un punto C, no puede salir otra de C a B. \\[10 pt]
Por ejemplo:
\begin{multicols}{2}
\center \includegraphics[scale=0.30]{Hola3.png}
\\[30 pt] \justify $$((b \wedge -d) \lor (d \wedge -b))\wedge(((c \wedge -g) $$ 
$$ \lor (g \wedge -c)) \wedge ((f \wedge -h) \lor (h \wedge -f))) $$
\end{multicols}
\end{frame}

%% Diapositiva 5
\begin{frame}{Representación de las interpretaciones}
Una representantación para los 3 puntos A, B, C es la siguiente : 
\\[20 pt]
\center \{'b': 1 , 'c': 0 , 'd': 0 , 'f' : 1 , 'g' : 1,  'h' : 0\}\\[20 pt]

\justify Si la interpretación de la letra proposicional es  '0' entonces significa que no hay una línea proveniente de algun punto a otro.\\[10 pt]
\justify Si la interpretación de la letra proposicional es  '1' entonces significa que hay una línea proveniente de algún punto a otro.
\end{frame}

%% Diapositiva 6
\begin{frame}{}
Dado lo anterior, la lista que se utilizó fue la siguiente : 
\\[20 pt]

\center Letras = \{'b','c','d','f','g','h'\}
\\[20 pt]

\justify Sin embargo como el problema consiste de n puntos, fue necesario utilizar la caracterización 'chr()' la cual a partir de un entero, genera un char, con el fin de representar las diferentes letras proposicionales.\\[10 pt]

\justify Es importante mencionar que como el problema se puede modelar con una matriz cuadrada en donde la diagonal es excluida en las letras proposicionales, entonces la cantidad de letras proposicionales es n(n) - n, donde n es el numero de puntos a conectar.\\[10 pt]

\end{frame}

%% Diapositiva 7
\begin{frame}{Representacion Grafica}
\begin{multicols}{2}
\center \includegraphics[scale=0.25]{imagen1.jpg}
\\
\justify Note que esta es la representacián gráfica de la interpretación mostrada anteriormente. En donde cada letra representa una posible línea que conecta dos puntos.\\[15 pt]
\justify \{'b': 1, 'f': 1, 'g': 1\}\\[10 pt]

\end{multicols}
\end{frame}

%% Diapositiva 8
\begin{frame}{Representacion para 4 puntos}
\justify Letras =  \{'È': 1 , 'É': 0 , 'Ê': 0 , 'Ë': 0 , 'Ì': 1 , 'Í': 0 , 'Î': 0 , 'Ï': 0 , 'Ð': 1 , 'Ñ': 1 , 'Ò': 0 , 'Ó': 0\}\\[15 pt]
\center \includegraphics[scale=0.25]{imagen4.jpg}

\end{frame}

%% Diapositiva 9
\begin{frame}{Representacion para 5 puntos}
\justify Letras = \{'È': 1, 'É': 0, 'Ê': 0, 'Ë': 0, 'Ì': 0, 'Í': 1, 'Î': 0, 'Ï': 0, 'Ð': 0, 'Ñ': 0, 'Ò': 1, 'Ó': 0, 'Ô': 0, 'Õ': 0, 'Ö': 0, '×': 1, 'Ø': 1, 'Ù': 0, 'Ú': 0, 'Û': 0\}\\[15 pt]
\center \includegraphics[scale=0.25]{imagen5.jpg}
\end{frame}
\end{document}







